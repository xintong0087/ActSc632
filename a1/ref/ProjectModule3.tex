\documentclass[11pt]{article}

\usepackage{amsfonts}
\usepackage{amssymb}
\usepackage{amsmath}
\usepackage{epsfig}
\usepackage{hyperref}

\setlength{\textheight}{8.85in}
\setlength{\textwidth}{6.75in}
\setlength{\topmargin}{0.0in}
\setlength{\headheight}{0.0in}
\setlength{\headsep}{0.0in}
\setlength{\oddsidemargin}{-.125in}
\setlength{\parskip}{2mm}
\setlength{\parindent}{0mm}

\begin{document}
\begin{center}
{\large \bf ACTSC 632 -- Individual Project 2 -- due on June 26} \\

\end{center}



For this project you will use data from the ``Motorcycle'' data set available in the {\tt insuranceData} package in which it is named {\tt dataOhlsson}.  The main goal of this project is to compute multipliers for the premium based on the different rating factors and corresponding levels deemed important in the model.

The current tariff  is given in the table at the end of the project description. It is based on four rating factors and was established some time ago.

%will consist of three main steps: computing multipliers for the frequency model, computing multipliers for the severity model, and then computing multipliers for the premium model. 

\begin{enumerate}
  
\item Download the data set {\tt dataOhlsson} and briefly discuss the data found in there (e.g., how many rating factors, what are the levels for each, how is the exposure determined, etc.). If there are any problems with the data, explain how you dealt with them.
\item Use only the rating factors contained in the current tariff. Compute the exposure (in policy years), the claim frequency and the average claim severity for each rating factor (i.e., for each possible value of each rating factor).
\item Use a relative Poisson glm to determine relativities for the claim frequency, using the current rating factors.
Provide a 95\% confidence interval for each relativity. Comment on the overall fit of this model to the data.
\item Use a Gamma glm (with log link function) to determine relativities for the severity, still using the current rating factors.
Provide a 95\% confidence interval for each relativity. Comment on the overall fit of this model to the data.
\item Assess whether rating factors for the policyholder's age and sex would have a significant impact. Include an interaction term. (I leave it up to you to decide how to handle the age variable, e.g., group it into intervals etc.).
\item Now combine your multiplier estimates for the frequency and severity data to get multipliers (and associated 95\% confidence intervals) for the premium overall and then propose a new tariff based on your analysis. Compare the results to the old tariff.
  \item Comment on any further analysis that should be considered before deciding on a final tariff. 
\end{enumerate}

Your report should be professionally organized, with a brief introduction. You are writing mainly for your colleagues in the pricing unit.

You should not include R code or output in the main part of the report. However you should upload to the LEARN dropbox all code used.

% on graduation

\begin{table*}
  \caption{Current tariff rating factors and relativities}
  \begin{center}
\begin{tabular}{|c|c|c|c|}
\hline
Rating Factor & Class & Description & Relativity \\
\hline
Zone & 1 & &7.768\\
&2 & & 4.227 \\
&3&& 1.336 \\
&4&& 1.000 \\
&5& & 1.734 \\
& 6&& 1.402 \\
&7&&1.402 \\
\hline
Motorcycle class & 1& & 0.625 \\
&2&& 0.769 \\
&3&& 1.000 \\
&4& & 1.406 \\
&5&& 1.875 \\
&6&& 4.062 \\
&7&&6.873 \\
\hline
Motorcycle Age & 1 & 0-1 years & 2.000\\
&2&2-4 years & 1.200\\
&3& $\ge 5$ years & 1.000 \\
\hline
Bonus Class & 1 & 1-2 & 1.250 \\
&2& 3-4 & 1.125\\
&3& 5-7 & 1.000 \\
\hline
\end{tabular}
\end{center}
\end{table*}

\iffalse
You can use Excel to do the calculations for the second-to-last question, but you should not have to manually input any numbers (i.e., estimates produced by R) in the spreadsheet. That is, you should output from R the numbers you need for your Excel spreadsheet calculations.

\fi


 
\end{document}













