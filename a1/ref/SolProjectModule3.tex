\documentclass[11pt]{article}

\usepackage{amsfonts}
\usepackage{amssymb}
\usepackage{amsmath}
\usepackage{epsfig}
\usepackage{hyperref}

\setlength{\textheight}{8.85in}
\setlength{\textwidth}{6.75in}
\setlength{\topmargin}{0.0in}
\setlength{\headheight}{0.0in}
\setlength{\headsep}{0.0in}
\setlength{\oddsidemargin}{-.125in}
\setlength{\parskip}{2mm}
\setlength{\parindent}{0mm}

\begin{document}
\begin{center}
{\large \bf ACTSC 632 -- Assignment 1 -- Solutions} \\

\end{center}

The purpose of these solutions is to show the results and analysis that you were asked to describe in your report. The solutions are not designed in the form of a report per se. The code used is in a separate file called {\tt motorcycle.txt} on LEARN.


%will consist of three main steps: computing multipliers for the frequency model, computing multipliers for the severity model, and then computing multipliers for the premium model. 

\begin{enumerate}
  
\item Download the data set {\tt dataOhlsson} and briefly discuss the data found in there (e.g., how many rating factors, what are the levels for each, how is the exposure determined, etc.). If there are any problems with the data, explain how you dealt with them. 

\noindent{\bf Solution:} as stated in the R documentation for this data set

\begin{quote}
The data for this case study comes from the former Swedish insurance company Wasa, and concerns partial casco insurance, for motorcycles this time. It contains aggregated data on all insurance policies and claims during 1994-1998; the reason for using this rather old data set is confidentiality; more recent data for ongoing business can not be disclosed.
\end{quote}

In this data set we have 6 rating factors given by agarald (age of driver),which goes from 0 to 99, kon (sex of driver), either M (male) or K (female), zon (geographical zone, going from 1 to 7, generally going from more to less urban), mcklass (class of the vehicle, with 7 possible classes, as determined by the EV ratio of engine power to vehicle weight), fordald (age of vehicle, a numerical value between 0 and 99), bonuskl (bonus class based on experience, going from 1 to 7: a new driver starts with bonus class 1; for each claim-free year the bonus class is increased by 1; after the first claim the bonus is decreased by 2; the driver can not return to class 7 with less than 6 consecutive claim free years). There are 64548 observations in this data set, with one for each driver observed. We see that for some observations the duration is 0. The exposure (or duration) is determined using policy-years, i.e., how long the contract was in effect for each driver.
For each observation we also have the number of claims (antskad) and the claim cost (skadkost).


\item Use only the rating factors contained in the current tariff. Compute the exposure (in policy years), the claim frequency and the average claim severity for each rating factor (i.e., for each possible value of each rating factor).

\noindent{\bf Solution:} After grouping the classes for the vehicle age into 3 groups  (0-2, 3-5, 6 and up) and the bonus class into three groups (1-2,3-4,5-7) we get

\begin{tabular}{lllll}
\hline
      &rating.factor class  & duration &n.claims & totcost (divided by 1000) \\
      \hline           
      &Zone    & 1  6205.3096    &  183 &  5539.963\\
           &Zone &    2 10103.0904   &   167 &   4811.166\\
           &Zone   &  3 11676.5726   &   123   & 2522.628\\
           &Zone    & 4 32628.4931   &   196   & 3774.629\\
           &Zone    & 5  1582.1123    &    9    & 104.739\\
           &Zone    & 6  2799.9452    &   18   &  288.045\\
           &Zone    & 7   241.2877      &  1     &  0.650\\
 &Vehicle class  &   1  5190.3507     &  46 &    993.062\\
 &Vehicle class   &  2  3990.1151     &  57   &  883.137\\
 &Vehicle class   &  3 21665.6794    &  166  &  5371.543\\
 &Vehicle class   &  4 11739.8821    &   98   & 2191.578\\
 &Vehicle class   &  5 13439.9260   &   149  &  3297.119\\
 &Vehicle class    & 6  8880.1342    &  175   & 4160.776\\
 &Vehicle class   &  7   330.7233      &  6   & 144.605\\
  & Vehicle age   &  1  4955.4027    &  126  &  4964.419\\
   &Vehicle age   &  2  9753.8109    &  145  &  5506.945\\
   &Vehicle age   &  3 50527.5972    &  426 &   6570.456\\
   &Bonus class   &  1 19893.3698   &   207  &  4558.072\\
   &Bonus class   &  2  9615.7644    &  121   & 3627.142\\
   &Bonus class   &  3 35727.6766 &     369   & 8856.606\\
\hline
\end{tabular}

Note that the last column shows the total claim cost (in thousands).
\item Use a relative Poisson glm to determine relativities for the claim frequency, using the current rating factors.
Provide a 95\% confidence interval for each relativity. Comment on the overall fit of this model to the data.

\noindent{\bf Solution:} we get

\begin{tabular}{lllll}
\hline
factor & factor class & multiplier & LB & UB\\
\hline
0&0&0.002345&0.001858&0.002959\\
Zone&1&5.156192&4.205633&6.321596\\
Zone&2&2.725123&2.215810&3.351503\\
Zone&3&1.708518&1.363530&2.140791\\
Zone&4&1.000000&1.000000&1.000000\\
Zone&5&0.906778&0.464785&1.769089\\
Zone&6&1.035100&0.638608&1.677762\\
Zone&7&0.727880&0.102011&5.193667\\
Vehicle class&1&1.478083&1.062390&2.056430\\
Vehicle class&2&2.103350&1.554868&2.845312\\
Vehicle class&3&1.000000&1.000000&1.000000\\
Vehicle class&4&1.321278&1.027817&1.698529\\
Vehicle class&5&2.045151&1.631045&2.564393\\
Vehicle class&6&3.979835&3.186922&4.970027\\
Vehicle class&7&3.311834&1.464417&7.489838\\
Vehicle age&1&3.239940&2.643934&3.970299\\
Vehicle age&2&1.894770&1.563719&2.295908\\
Vehicle age&3&1.000000&1.000000&1.000000\\
Bonus class&1&1.275967&1.067856&1.524635\\
Bonus class&2&1.443011&1.171850&1.776917\\
Bonus class&3&1.000000&1.000000&1.000000\\
\hline
\end{tabular}

With a deviance of 360.2168 on 389 degrees of freedom, the model seems a reasonable fit ($p$-value of 0.8495).

Note that I also accepted answers based on the quasi-Poisson family where a dispersion parameter is estimated, causing the CIs to be different from the above.

\item Use a Gamma glm (with log link function) to determine relativities for the severity, still using the current rating factors.
Provide a 95\% confidence interval for each relativity. Comment on the overall fit of this model to the data.

\noindent {\bf Solution:} we get the multipliers

\begin{tabular}{lllll}
\hline
factor & factor class & multiplier & LB & UB\\
\hline
0&0&1.570e+04&1.132e+04&2.177e+04\\
Zone&1&1.300e+00&9.679e-01&1.747e+00\\
Zone&2&1.370e+00&1.019e+00&1.842e+00\\
Zone&3&9.364e-01&6.771e-01&1.295e+00\\
Zone&4&1.000e+00&1.000e+00&1.000e+00\\
Zone&5&9.634e-01&3.658e-01&2.537e+00\\
Zone&6&7.845e-01&3.878e-01&1.587e+00\\
Zone&7&1.765e-02&1.048e-03&2.975e-01\\
Vehicle class&1&7.459e-01&4.661e-01&1.194e+00\\
Vehicle class&2&6.673e-01&4.302e-01&1.035e+00\\
Vehicle class&3&1.000e+00&1.000e+00&1.000e+00\\
Vehicle class&4&7.976e-01&5.560e-01&1.144e+00\\
Vehicle class&5&8.330e-01&6.010e-01&1.155e+00\\
Vehicle class&6&1.035e+00&7.504e-01&1.427e+00\\
Vehicle class&7&1.433e+00&4.354e-01&4.716e+00\\
Vehicle age&1&2.556e+00&1.910e+00&3.420e+00\\
Vehicle age&2&2.345e+00&1.774e+00&3.101e+00\\
Vehicle age&3&1.000e+00&1.000e+00&1.000e+00\\
Bonus class&1&8.356e-01&6.455e-01&1.082e+00\\
Bonus class&2&1.031e+00&7.664e-01&1.386e+00\\
Bonus class&3&1.000e+00&1.000e+00&1.000e+00\\
\hline
\end{tabular}

The severity model doesn't seem to be a very good fit. We reject the hypothesis that the data comes from this model based on the residual deviance, given by 351 on 164 degrees of freedom, as its corresponding $p$-value is 1.13842e-15.

\item Assess whether rating factors for the policyholder's age and sex would have a significant impact. Include an interaction term. (I leave it up to you to decide how to handle the age variable, e.g., group it into intervals etc.).

\noindent {\bf Solution:} we have combined the ages into 2 groups given by the breaks 0, 30, 100. When including age and sex with an interaction term, for the frequency the deviance goes to 742.76  on 1277 degrees of freedom. The LRT statistic we get

742.76- 360.22 =382.55 on 1277-389=888 degrees of freedom. The corresponding $p$-value is 1, suggesting that the simpler model is a better fit.

For the severity however, the LRT statistic is given by 253.12 on 124 degrees of freedom, with corresponding $p$-value given by $7.09 \times 10^{-11}$. Hence in this case, the age and sex do appear to provide a model that has a better fit. We also see that for  both the frequency and the severity models, while sex is not a significant factor, the interaction term between age and sex and the coefficient for the age group 0-30 are both significant.
% dispersion model 1 freq is 1.410456

%(Dispersion parameter for Gamma family taken to be 1.66371)

%    Null deviance: 1001.13  on 415  degrees of freedom
%Residual deviance:  767.11  on 388  degrees of freedom


%(Dispersion parameter for Gamma family taken to be 2.041871)

%    Null deviance: 540.16  on 180  degrees of freedom
%Residual deviance: 351.11  on 164  degrees of freedom
%AIC: 15140

\item Now combine your multiplier estimates for the frequency and severity data to get multipliers (and associated 95\% confidence intervals) for the premium overall and then propose a new tariff based on your analysis. Compare the results to the old tariff.

\begin{tabular}{llllll}
\hline
factor & factor class& new multiplier &LB &UB & old multiplier \\
\hline
&&36.81174&2.465e+01&54.9782&0.000\\
Zone&1& 6.70508&4.684e+00& 9.5988&7.768\\
Zone&2& 3.73268&2.601e+00& 5.3568&4.227\\
Zone&3& 1.59982&1.078e+00& 2.3747&1.336\\
Zone&4& 1.00000&1.000e+00& 1.0000&1.000\\
Zone&5& 0.87358&2.693e-01& 2.8336&1.734\\
Zone&6& 0.81205&3.456e-01& 1.9079&1.402\\
Zone&7& 0.01285&4.117e-04& 0.4011&1.402\\
Vehicle class&1& 1.10257&6.206e-01& 1.9588&0.625\\
Vehicle class&2& 1.40354&8.237e-01& 2.3915&0.769\\
Vehicle class&3& 1.00000&1.000e+00& 1.0000&1.000\\
Vehicle class&4& 1.05390&6.790e-01& 1.6359&1.406\\
Vehicle class&5& 1.70368&1.145e+00& 2.5345&1.875\\
Vehicle class&6& 4.11778&2.786e+00& 6.0857&4.062\\
Vehicle class&7& 4.74583&1.120e+00&20.1095&6.873\\
Vehicle age&1& 8.28051&5.805e+00&11.8111&2.000\\
Vehicle age&2& 4.44415&3.167e+00& 6.2372&1.200\\
Vehicle age&3& 1.00000&1.000e+00& 1.0000&1.000\\
Bonus class&1& 1.06615&7.792e-01& 1.4587&1.250\\
Bonus class&2& 1.48750&1.036e+00& 2.1367&1.125\\
Bonus class&3& 1.00000&1.000e+00& 1.0000&1.000\\
\hline
\end{tabular}

We see that the new tariff is much higher than the old one for Vehicle Age 1 and 2. We also see that for Zone 5 to 7, the new tariff suggests to lower the premium compared to the baseline of Zone 4, while for the old tariff they were all higher than the baseline. The number of claims for these 3 zones is very small though, so it seems like it might be best to not make such an important change based on such a small sample.

  \item Comment on any further analysis that should be considered before deciding on a final tariff. 
  
  \noindent{\bf Solution:} we saw that, especially for the severity, the age and sex seem to be significant factors that should be considered. On the other hand, looking at the results for coefficients in each of the frequency and severity models (reproduced below), we see that Zones 5,6,7 (note that although the results below are based on the ordering of levels by decreasing order of duration, and therefore do not necessarily correspond to the original numbering, for these 3 particular classes are actually the same as in the original ordering) do not seem to be significant, and should therefore probably be combined with the baseline of Zone 4. Finally, we should explore the use of models other than the gamma distribution for the severity, to see if a better fit can be obtained.
  
  \begin{verbatim}
  FREQUENCY
  Coefficients:
             Estimate Std. Error z value Pr(>|z|)    
(Intercept)  -6.05548    0.11870 -51.016  < 2e-16 ***
zon2          0.53563    0.11508   4.655 3.25e-06 ***
zon3          1.00251    0.10556   9.497  < 2e-16 ***
zon4          1.64020    0.10397  15.776  < 2e-16 ***
zon5          0.03450    0.24641   0.140 0.888657    
zon6         -0.09786    0.34098  -0.287 0.774122    
zon7         -0.31762    1.00258  -0.317 0.751394    
mcklass2      0.71547    0.11543   6.198 5.71e-10 ***
mcklass3      0.27860    0.12814   2.174 0.029697 *  
mcklass4      1.38124    0.11336  12.185  < 2e-16 ***
mcklass5      0.39075    0.16848   2.319 0.020384 *  
mcklass6      0.74353    0.15415   4.823 1.41e-06 ***
mcklass7      1.19750    0.41635   2.876 0.004025 ** 
motorage_gr2  0.63910    0.09797   6.523 6.89e-11 ***
motorage_gr3  1.17555    0.10372  11.334  < 2e-16 ***
bonuskl2      0.24370    0.09084   2.683 0.007303 ** 
bonuskl3      0.36673    0.10620   3.453 0.000554 ***
---
Signif. codes:  0 �***� 0.001 �**� 0.01 �*� 0.05 �.� 0.1 � � 1
  \end{verbatim}
  
   \begin{verbatim}
  SEVERITY
  Coefficients:
             Estimate Std. Error t value Pr(>|t|)    
(Intercept)   9.66130    0.16671  57.951  < 2e-16 ***
zon2         -0.06573    0.16543  -0.397   0.6916    
zon3          0.31461    0.15108   2.082   0.0389 *  
zon4          0.26267    0.15066   1.743   0.0831 .  
zon5         -0.24269    0.35946  -0.675   0.5005    
zon6         -0.03730    0.49412  -0.075   0.9399    
zon7         -4.03682    1.44106  -2.801   0.0057 ** 
mcklass2     -0.18268    0.16657  -1.097   0.2744    
mcklass3     -0.22610    0.18413  -1.228   0.2212    
mcklass4      0.03407    0.16392   0.208   0.8356    
mcklass5     -0.29311    0.23996  -1.221   0.2237    
mcklass6     -0.40453    0.22397  -1.806   0.0727 .  
mcklass7      0.35976    0.60776   0.592   0.5547    
motorage_gr2  0.85249    0.14250   5.982 1.34e-08 ***
motorage_gr3  0.93835    0.14857   6.316 2.42e-09 ***
bonuskl2     -0.17965    0.13165  -1.365   0.1743    
bonuskl3      0.03037    0.15121   0.201   0.8411    
\end{verbatim}
\end{enumerate}


% on graduation


\iffalse
You can use Excel to do the calculations for the second-to-last question, but you should not have to manually input any numbers (i.e., estimates produced by R) in the spreadsheet. That is, you should output from R the numbers you need for your Excel spreadsheet calculations.

\fi


 
\end{document}













