\documentclass[aspectratio=169,professionalfonts,mathserif,handout]{beamer}

\usepackage{pgf,tikz}
\usetikzlibrary{arrows,automata}
\usepackage{dsfont}
\usepackage{amsmath}
\usepackage{amssymb}
\usepackage{amsthm}
\usepackage{eulervm}
\usepackage{wasysym}
\usepackage[scaled]{helvet}
\usepackage[english]{babel}
\usepackage[T1]{fontenc}
\usepackage{hhline}
\usepackage{cancel}
\usepackage{comment}
\usepackage{booktabs}
\usepackage{multirow}
\usepackage{bm}
\usepackage{array}
\usepackage{ulem}
\usepackage[absolute,overlay]{textpos}
\usepackage{algorithm2e}



\usetikzlibrary{shapes,patterns,positioning,shapes.misc}
\renewcommand{\arraystretch}{1.2}
\frenchspacing

\newcommand\itemshape[1]{%
  \setbeamertemplate{itemize item}[#1]%
  \usebeamertemplate{itemize item}%
}

%-----------------------------------------------------------------------------%
% tikz
%-----------------------------------------------------------------------------%



\tikzset{cross/.style={cross out, draw=black, minimum size=2*(#1-\pgflinewidth), inner sep=0pt, outer sep=0pt},
%default radius will be 1pt. 
cross/.default={3pt}}


%-----------------------------------------------------------------------------%
% Colors
%-----------------------------------------------------------------------------%

% Black & White:

\definecolor{White}{rgb}{1,1,1}
\definecolor{LightGray}{rgb}{.91,.91,.91}
\definecolor{MediumGray}{rgb}{.75,.75,.75}
\definecolor{DarkGray}{rgb}{.4,.4,.4}
\definecolor{VeryDarkGray}{rgb}{.2,.2,.2}
\definecolor{Black}{rgb}{0,0,0}

% Purples:

\definecolor{RedPurple}{rgb}{.9,.0,.7}
\definecolor{LightPurple}{rgb}{.45,.0,.45}
\definecolor{DarkPurple}{rgb}{.1,.0,.3}
\definecolor{LightMagenta}{rgb}{0.95,.9,1}

% Blues:

\definecolor{Blue}{rgb}{0,0,0.9}
\definecolor{Navy}{rgb}{0,0,.5}
\definecolor{DarkBlue}{rgb}{0,0.1,0.3}

% Greens:

\definecolor{DarkGreen}{rgb}{.0,.2,.0}
\definecolor{Green}{rgb}{.0,.4,.0}

% Browns:

\definecolor{YellowBrown}{rgb}{1,1,0.7}
\definecolor{Cream}{rgb}{.9,.9,.75}
\definecolor{DarkerCream}{rgb}{.7,.7,.63}
\definecolor{Brown}{rgb}{0.2,0.25,0.1}
\definecolor{Coal}{rgb}{0.1,0.1,0.08}

% Other

\definecolor{CookieDough}{HTML}{ccccaa}
\definecolor{Clay}{HTML}{757561}
\definecolor{Licorice}{HTML}{21211c}
\definecolor{Red}{rgb}{0.8,0,0}

\colorlet{Back}{White}
\colorlet{Text}{Black}
\colorlet{TitleText}{DarkBlue}
\colorlet{Highlight}{DarkBlue}

\newcommand{\dg}[1]{{\color{Green}{#1}}}
\newcommand{\ora}[1]{{\color{orange}{#1}}}
\newcommand{\bl}[1]{{\color{Blue}{#1}}}
\newcommand{\red}[1]{{\color{Red}{#1}}}
\newcommand{\purple}[1]{{\color{DarkPurple}{#1}}}
\newcommand{\grey}[1]{{\color{MediumGray}{#1}}}

%\colorlet{Back}{Black!95}
%\colorlet{Text}{White}
%\colorlet{TitleText}{Blue}
%\colorlet{Highlight}{yellow}

%-----------------------------------------------------------------------------%
% Macros
%-----------------------------------------------------------------------------%

\newcommand{\tsp}{\mspace{1mu}}
\newcommand{\htsp}{\mspace{0.5mu}}
\newcommand{\op}[1]{\operatorname{#1}}
\newcommand{\tr}{\operatorname{Tr}}
\newcommand{\rank}{\operatorname{rank}}
\renewcommand{\t}{{\scriptscriptstyle\mathsf{T}}}
\newcommand{\reg}[1]{\mathsf{#1}}
\newcommand{\im}{\op{im}}
\newcommand{\fid}{\op{F}}
%\renewcommand{\vec}{\op{vec}}
\newcommand{\I}{\mathds{1}}

\newcommand{\abs}[1]{\lvert #1 \rvert}
\newcommand{\bigabs}[1]{\bigl\lvert #1 \bigr\rvert}
\newcommand{\Bigabs}[1]{\Bigl\lvert #1 \Bigr\rvert}
\newcommand{\biggabs}[1]{\biggl\lvert #1 \biggr\rvert}
\newcommand{\Biggabs}[1]{\Biggl\lvert #1 \Biggr\rvert}

\newcommand{\ip}[2]{\langle #1 , #2\rangle}
\newcommand{\bigip}[2]{\bigl\langle #1, #2 \bigr\rangle}
\newcommand{\Bigip}[2]{\Bigl\langle #1, #2 \Bigr\rangle}
\newcommand{\biggip}[2]{\biggl\langle #1, #2 \biggr\rangle}
\newcommand{\Biggip}[2]{\Biggl\langle #1, #2 \Biggr\rangle}

\newcommand{\ceil}[1]{\lceil #1 \rceil}
\newcommand{\bigceil}[1]{\bigl\lceil #1 \bigr\rceil}
\newcommand{\Bigceil}[1]{\Bigl\lceil #1 \Bigr\rceil}
\newcommand{\biggceil}[1]{\biggl\lceil #1 \biggr\rceil}
\newcommand{\Biggceil}[1]{\Biggl\lceil #1 \Biggr\rceil}

\newcommand{\floor}[1]{\lfloor #1 \rfloor}
\newcommand{\bigfloor}[1]{\bigl\lfloor #1 \bigr\rfloor}
\newcommand{\Bigfloor}[1]{\Bigl\lfloor #1 \Bigr\rfloor}
\newcommand{\biggfloor}[1]{\biggl\lfloor #1 \biggr\rfloor}
\newcommand{\Biggfloor}[1]{\Biggl\lfloor #1 \Biggr\rfloor}

\newcommand{\norm}[1]{\lVert\tsp #1 \tsp\rVert}
\newcommand{\bignorm}[1]{\bigl\lVert\tsp #1 \tsp\bigr\rVert}
\newcommand{\Bignorm}[1]{\Bigl\lVert\tsp #1 \tsp\Bigr\rVert}
\newcommand{\biggnorm}[1]{\biggl\lVert\tsp #1 \tsp\biggr\rVert}
\newcommand{\Biggnorm}[1]{\Biggl\lVert\tsp #1 \tsp\Biggr\rVert}




%\usepackage{amsfonts}
%\usepackage{graphicx}
%\usepackage{epsfig}
%\usepackage{subfigure}
%\usepackage{amssymb}% sophisticated mathematical symbols with amstex
%\usepackage{mathtools}% fix amsmath deficiencies
%\usepackage[T1]{fontenc}
%\usepackage{bm}% for bold math symbols
%\usepackage{bbm}% only for indicator functions
%\useoutertheme{default}
%\beamersetuncovermixins{\opaqueness<1>{25}}{\opaqueness<2->{15}}

\def\ba{{\mathbf a}}
\def\bd{{\mathbf d}}
%\def\bi{{\mathbf i}}
\def\bu{{\mathbf u}}
\def\bx{{\mathbf x}}
\def\bU{{\bf U}}
\def\bV{{\mathbf V}}
\def\bz{{\mathbf z}}
\def\bW{{\mathbf W}}
\def\bv{{\mathbf v}}
%\def\by{{\mathbf y}}
\def\bY{{\mathbf Y}}
\def\bw{{\mathbf w}}
\def\bX{{\mathbf X}}
\def\QMC{{\rm QMC}}
\def\bh{{\mathbf h}}
\def\MC{{\rm MC}}
\def\tra{{\mathrm T }}
\def\bz {\mathbf{z}}
\def\btpsi {\tilde{\boldsymbol{\psi}}}
\def\btv {\tilde{\mathbf{v}}}
\def\IR {\mathbb{R}}
\def\IP {\mathbb{P}}
\newcommand*{\IE}{\mathbbm{E}}
\newcommand*{\D}{\operatorname{D}}
\newcommand*{\Bcal}{\mathcal{B}}
\newcommand*{\Exp}{\operatorname{Exp}}
\newcommand*{\psii}{{\psi^{-1}}}
\newcommand*{\LS}{\mathcal{LS}}
\newcommand*{\LSi}{\LS^{-1}}

%\xdefinecolor{mypurple}{rgb}{0.6,0,0.7}
%\xdefinecolor{darkgreen}{rgb}{0,0.6,0}
%\xdefinecolor{MidYellow}{rgb}{1,0.5,0.2}

%\newcommand{\magenta}[1]{{\color{magenta}{#1}}}
%\newcommand{\gr}[1]{{\color{darkgreen}{#1}}}
%\newcommand{\MidYellow}[1]{{\color{MidYellow}{#1}}}
%\newcommand{\gray}[1]{{\color{gray}{#1}}}

\def\apause{\pause}
%\def\apause{}

\newenvironment{mylist}[1]{
  \begin{list}{}{
      \setlength{\leftmargin}{#1}
      \setlength{\rightmargin}{0mm}
      \setlength{\labelsep}{2mm}
      \setlength{\labelwidth}{8mm}
      \setlength{\itemsep}{0mm}}}{\end{list}}

\newcommand{\reference}[1]{\textcolor{Navy}{\footnotesize \sc #1}}

\newcommand{\highlight}[1]{\textcolor{Highlight}{\bf #1}}

%-----------------------------------------------------------------------------%
%Ruodu's stuff
%-----------------------------------------------------------------------------%


\def\fsd{\succcurlyeq_{\hspace{- 0.4 mm} _{fsd}}}
\def\sfsd{\succ_{\hspace{-0.4 mm} _{fsd}}}
\def\ssdmu{\succcurlyeq_{\hspace{- 0.4 mm} _{SSD\(\mu\)}}}
\def\ssdnu{\succcurlyeq_{\hspace{- 0.4 mm} _{SSD\(\nu\)}}}
\def\ssdP{\succcurlyeq_{\hspace{- 0.4 mm} _{SSD\(P\)}}}
\def\sssdmu{\succ_{\hspace{- 0.4 mm} _{SSD\(\mu\)}}}
\def\sssdnu{\succ_{\hspace{- 0.4 mm} _{SSD\(\nu\)}}}
\def\sssdP{\succ_{\hspace{- 0.4 mm} _{SSD\(P\)}}}
\def\Ito{It\^o}

\newcommand{\be}{\begin{eqnarray}}
\newcommand{\ee}{\end{eqnarray}}
\newcommand{\by}{\begin{eqnarray*}}
\newcommand{\ey}{\end{eqnarray*}}
\newcommand{\bi}{\begin{itemize}}
\newcommand{\ei}{\end{itemize}}

\DeclareMathOperator*{\argmax}{\arg\max}
\newtheorem{remark}[theorem]{Remark}
\newtheorem{assumption}[theorem]{Assumption}
\newtheorem{proposition}[theorem]{Proposition}

\setbeamercolor{yellow}{fg=black,bg=yellow}
\setbeamercolor{lightyellow}{fg=black,bg=yellow!40}
\setbeamercolor{orange}{fg=black,bg=orange}
\setbeamercolor{lightorange}{fg=black,bg=orange!40}
\setbeamercolor{lavender}{fg=black,bg= LavenderBlush!20!Lavender}

\newenvironment{colorblock}[2]
{\setbeamercolor{item}{fg=#1,bg=#1}\begin{beamerboxesrounded}[upper=#1,lower=#2,shadow=true]}
{\end{beamerboxesrounded}}

\definecolor{darkblue}{rgb}{0.1,0.1,0.6}
\definecolor{darkgreen}{rgb}{0.1,0.6,0.1}
\definecolor{fond}{RGB}{240,240,240}

\newcommand{\bluearrow}{\color{darkblue}{$\blacktriangleright$}\color{black}}
\newcommand{\redarrow}{\color{red}{$\blacktriangleright$}\color{black}}
\newcommand{\greenarrow}{\color{green}{$\blacktriangleright$}\color{black}}

\renewcommand{\(}{\left(}
\renewcommand{\)}{\right)}
\renewcommand{\[}{\left[}
\renewcommand{\]}{\right]}
\newcommand{\w}{\widehat}


%-----------------------------------------------------------------------------%
% Beamer stuff. Default theme -- everything needed is here.
%-----------------------------------------------------------------------------%

\usetheme{default}
\usefonttheme[onlymath]{serif}

% Aspect ratio is set to 16:9
% Total page width is 16cm

\setbeamersize{%
  text margin left=8mm,
  text margin right=8mm}

\addtolength{\headsep}{2.5mm}

\setbeamertemplate{navigation symbols}{}

\setbeamerfont{block title}{series=\bfseries}
\setbeamerfont{titlelike}{series=\bfseries}
\setbeamercolor{structure}{fg=Black}

\setbeamercolor*{titlelike}{fg=TitleText,bg=}
\setbeamercolor{normal text}{fg=Text,bg=Back}

\newcommand{\mytitle}[1]{\vspace*{-1mm}%
  \centerline{\highlight{\Large #1}}\vspace*{3mm}}

\newenvironment{slidebox}{%
  \begin{minipage}[c][7.5cm][t]{14.4cm}\raggedright}{%
  \end{minipage}}

\newenvironment{halfpage}{%
  \begin{minipage}[c][6.5cm][t]{6.4cm}\raggedright}{%
  \end{minipage}}

%-----------------------------------------------------------------------------%
\title{ACTSC 632 Week 4}
\author{Christiane Lemieux}
\date{\today}


\begin{document}


%-----------------------------------------------------------------------------%

\begin{frame}
  \begin{slidebox}
    
    \vspace{7mm}
    \begin{center}
      \begin{tikzpicture}

        \onslide<1>{%
        \node at (0,0) {%
          \makebox(0,0){\includegraphics[width=5.5cm]{UWlogo.eps}}};}

        \onslide<1>{%
        \node at (10.5,0) {%
          \makebox(0,0){ACTSC 632 Spring 2023%
            \rule[-3mm]{0mm}{6mm}
            \rule{6mm}{0mm}}};}
        
      \end{tikzpicture}
    \end{center}
    
    \vspace{6mm}
    
    \begin{center}
      \onslide<1>{%
      \highlight{\large Data Science with Actuarial Applications}\\[3mm]}
       \onslide<1>{\Huge {\color{darkblue} Week 6}}\\[6mm]
          \onslide<1>{\large %
        {\large Xintong Li}\\[3mm]
        Department of Statistics and Actuarial Science}
    \end{center}
    
\end{slidebox}\end{frame}

\begin{frame}\begin{slidebox}
\mytitle{Last Week}

\begin{itemize}
    \item $Y$ is a key ratio: claim frequency or claim severity
    \pause
    \item $X$ is a vector of rating factors, modeled as categorical variables (can still use dummy binary variables to implement this in R)
    \pause
    \item terminology: duration, claim frequency, claim severity, rating cell
    \pause
    \item 3 key assumptions
    \pause
    \item ${\rm E}(Y) = \mu$, ${\rm Var}(Y) = \sigma^2/w$
    \pause
    \item Moped example (in R)
\end{itemize}
\end{slidebox}\end{frame}

\begin{frame}\begin{slidebox}
\mytitle{Today}

\begin{itemize}
    \item Introduce the idea of multiplicative models
    \pause
    \item Basic model for claim frequency
    \pause
    \item Basic model for claim severity
    \pause
    \item How will we use GLMs

\end{itemize}
\end{slidebox}\end{frame}

\begin{frame}\begin{slidebox}
\mytitle{Note on Multiplicative Models}
\begin{tikzpicture}[domain=0:2]
\draw[blue, very thick] (0,0)  rectangle (14,6) ; 
\end{tikzpicture}
\end{slidebox}\end{frame}

\begin{frame}\begin{slidebox}
\mytitle{2.4.1 Basic model for claim frequency}
 \begin{tikzpicture}[domain=0:2]
\draw[blue, very thick] (0,0)  rectangle (14,6) ; 
\end{tikzpicture}
\end{slidebox}\end{frame}

\begin{frame}\begin{slidebox}
\mytitle{Reproductive property of the relative Poisson distribution}
 \begin{tikzpicture}[domain=0:2]
\draw[blue, very thick] (0,0)  rectangle (14,6) ; 
\end{tikzpicture}
\end{slidebox}\end{frame}


\begin{frame}\begin{slidebox}
\mytitle{2.4.2 Basic model for claim severity}
 \begin{tikzpicture}[domain=0:2]
\draw[blue, very thick] (0,0)  rectangle (14,6) ; 
\end{tikzpicture}

For both the frequency and the severity, the impact of the rating factors will be incorporated via $\mu_i$, using the theory of GLMs.
\end{slidebox}\end{frame}

\begin{frame}\begin{slidebox}
\mytitle{2.5 The basics of pricing with GLMs}

\begin{itemize}
    \item \highlight{Goal:} determine how key ratio $Y$ varies with rating factors
    \pause
    \item Q: Why not use multiple linear regression?
    \pause
    \begin{itemize}
        \item (A) Assumption of normally distributed error may not be reasonable, e.g., for nb of claims (which is discrete) or claim size (which is skewed and $>0$)
        \pause
        \item (B) modeling $Y$ as a linear (additive) fct of the rating factors clashes with our intuition to use a multiplicative model
        \pause
        \item (C) error term forced to have constant variance
    \end{itemize}
    \pause
    \item \highlight{Advantages of GLM for pricing}
    \pause
    \begin{enumerate}
        \item has theory that can be used to estimate std error, build CIs, do model selection
        \pause
        \item used in many areas, so can benefit for developments elsewhere
        \pause
        \item std software is available for fitting 
    \end{enumerate}
\end{itemize}
\end{slidebox}\end{frame}

\begin{frame}\begin{slidebox}
\mytitle{Quick review of GLMs}
 \begin{tikzpicture}[domain=0:2]
\draw[blue, very thick] (0,0)  rectangle (14,6) ; 
\end{tikzpicture}
\end{slidebox}\end{frame}

\begin{frame}\begin{slidebox}
\mytitle{What is next}

\begin{itemize}
\item Next we discuss how \highlight{Exponential Dispersion Models (EDM)} can be used to address (A) (\grey{Assumption of normally distributed error may not be reasonable})
in the context of non-life insurance pircing.
\pause
\item Then we'll review how the flexibility in choosing the \highlight{link function} can be used to address (B) (\grey{modeling $Y$ as a linear (additive) fct of the rating factors}) and allow us to use a multiplicative model
\end{itemize}
\end{slidebox}\end{frame}

\begin{frame}\begin{slidebox}
\mytitle{2.5.1 Exponential Dispersion Models}
 \begin{tikzpicture}[domain=0:2]
\draw[blue, very thick] (0,0)  rectangle (14,6) ; 
\end{tikzpicture}
\end{slidebox}\end{frame}

\begin{frame}\begin{slidebox}
\mytitle{Recap of Week 6 -- Lecture 1}
\begin{itemize}
    \item \highlight{Basic claim frequency model}: use
    $$
\mathbb{P}\left(  Y_{i}=y_{i}\right)  =\frac{\left(  w_{i}\mu_{i}\right)
^{w_{i}y_{i}}e^{-w_{i}\mu_{i}}}{k!}\text{,} \qquad y_{i}\in\left\{  0,\frac{1}{w_{i}},\frac{2}{w_{i}},...\right\}  
$$
\pause
\item \highlight{Basic claim severity model}: use
\begin{align*}
f_{Y_{i}}\left(  y\right)   &  =\frac{\left(  \frac{\mu_{i}}{\xi_{i}}%
w_{i}\right)  ^{\frac{w_{i}\left(  \mu_{i}\right)  ^{2}}{\xi_{i}}}%
y^{\frac{w_{i}\left(  \mu_{i}\right)  ^{2}}{\xi_{i}}-1}e^{-\frac{w_{i}\mu_{i}%
}{\xi}y}}{\Gamma\left(  \frac{w_{i}\left(  \mu_{i}\right)  ^{2}}{\xi_{i}%
}\right)  }, \qquad y_i>0\nonumber\\
\end{align*}

\end{itemize}
\end{slidebox}\end{frame}

\begin{frame}\begin{slidebox}
\mytitle{Recap of Week 6 -- Lecture 1}
\begin{itemize}
\item $Y_i$ is EDM if pdf/pmf given by 
$$
f_{Y_{i}}\left(  y_{i},\theta_{i},\phi\right)  =\exp\left\{  \frac{y_{i}%
\theta_{i}-b\left(  \theta_{i}\right)  }{\frac{\phi}{w_{i}}}+c\left(
y_{i},\phi,w_{i}\right)  \right\}  \text{,} \label{EDM}%
$$
\pause
\item Forgot to provide notation $\eta = \beta_0 + \beta_1 x_{1}+\ldots + \beta_p x_p$ for the linear predictor
\end{itemize}
\end{slidebox}\end{frame}

\begin{frame}\begin{slidebox}
\mytitle{Additional properties of EDM}
 \begin{tikzpicture}[domain=0:2]
\draw[blue, very thick] (0,0)  rectangle (14,6) ; 
\end{tikzpicture}
\end{slidebox}\end{frame}

\begin{frame}\begin{slidebox}
\mytitle{Checking that relative Poisson distribution is EDM}
 \begin{tikzpicture}[domain=0:2]
\draw[blue, very thick] (0,0)  rectangle (14,6) ; 
\end{tikzpicture}
\end{slidebox}\end{frame}

\begin{frame}\begin{slidebox}
\mytitle{Checking that Gamma model is an EDM}
 \begin{tikzpicture}[domain=0:2]
\draw[blue, very thick] (0,0)  rectangle (14,6) ; 
\end{tikzpicture}
\end{slidebox}\end{frame}

\begin{frame}\begin{slidebox}
\mytitle{Reproductive property of EDM family}

\highlight{Result:} If $Y_1$ and $Y_2$ are two independent rv's from the same EDM family (i.e., same $b(\cdot)$, same $\mu$ and same $\phi$) but with possibly different weights $w_1$ and $w_2$ then $Y = \frac{w_1Y_1+w_2Y_2}{w_1+w_2}$ is in same EDM family but with weight $w_1+w_2$.
\end{slidebox}\end{frame}

\begin{frame}\begin{slidebox}
\mytitle{2.5.2 Link Function}

\begin{itemize}
 \item Previous topic helped us identify a rich class of models for $Y$ that are useful in the context of GLMs
 \pause
 \item The \highlight{link function} provides flexibility to model how the
response (i.e., the key ratio $Y_{i}$) relates to the rating factors. 
\pause
\item  This is done by
imposing a \highlight{relationship} between the mean response $\mathbb{E}\left[
Y_{i}\right]  =\mu_{i}$ and the set of rating factors.
\end{itemize}

\end{slidebox}\end{frame}

\begin{frame}\begin{slidebox}
\mytitle{Example with one rating factor}
\begin{itemize}
    

\item Consider a tariff model with \dg{only one rating factor} taking possible values
$\left\{  a,b,c\right\}$.
\pause
\item One category will be the \dg{base category} for the rating factor (say,
category $a$) and \highlight{2 binary variables} will be created to indicate \bl{whether the
category is $b$ or not}, and \ora{whether the category is $c$ or not}. \pause Hence we get the linear predictor


$$
\eta_{i}=\beta_{0}+\bl{\beta_{1}}x_{i1}+\ora{\beta_{2}}x_{i2}\text{,}%
$$
where \bl{$x_{i1}=1$ if tariff cell $i$ has category $b$} for rating
factor (or otherwise 0) and \ora{$x_{i2}=1$ if tariff cell $i$ has category $c$} for
its rating factor (or otherwise 0). \pause Hence

\begin{center}
\begin{tabular}
[c]{|ccc|}\hline
\multicolumn{1}{|c|}{Tariff cell $i$} & Rating factor & $\eta_{i}%
$\\\hline\hline
\multicolumn{1}{|c|}{1} & $\left(  a\right)  $ & $\beta_{0}$\\
\multicolumn{1}{|c|}{2} & $\left(  b\right)  $ & $\beta_{0}+\bl{\beta_{1}}$\\
\multicolumn{1}{|c|}{3} & $\left(  c\right)  $ & $\beta_{0}+\ora{\beta_{2}}$\\\hline
\end{tabular}
\end{center}
\end{itemize}
\end{slidebox}\end{frame}

\begin{frame}\begin{slidebox}


\mytitle{Example with two rating factors}
Consider now a tariff model with two rating factors:
\pause
\begin{itemize}
\item \bl{Rating factor 1 takes on two possible values} $\left\{  a,b\right\}  $;
\pause
\item \ora{Rating factor 2 takes on three possible values} $\left\{  c,d,e\right\}
$.
\pause
\item Create \bl{1 binary variable} for Rating
factor 1 (assuming base category is $a$)
\pause
\item Create \ora{2 binary variables} for Rating factor 2 (assuming base category is $c$).
\end{itemize}
\pause
Hence, we define
$$
\eta_{i}=\beta_{0}+\bl{\beta_{1}}x_{i1}+\ora{\beta_{2}}x_{i2}+\ora{\beta_{3}}x_{i3}\text{,}%
$$
\pause where

\begin{itemize}
\item \bl{$x_{i1}=1$} if tariff cell $i$ has category $b$ for its first rating
factor (or otherwise 0);
\pause
\item \ora{$x_{i2}=1$} if tariff cell $i$ has category $d$ for its second rating
factor (or otherwise 0)
\pause
\item \ora{$x_{i3}=1$} if tariff cell $i$ has category $e$ for its second rating
factor (or otherwise 0)
\end{itemize}
\end{slidebox}\end{frame}

\begin{frame}\begin{slidebox}


\begin{center}
\begin{tabular}
[c]{|ccc|}\hline
\multicolumn{1}{|c|}{Tariff cell $i$} & Rating factors & $\eta_{i}%
$\\\hline\hline
\multicolumn{1}{|c|}{1} & $\left(  a,c\right)  $ & $\beta_{0}$\\
\multicolumn{1}{|c|}{2} & $\left(  a,\ora{d}\right)  $ & $\beta_{0}+\ora{\beta_{2}}$\\
\multicolumn{1}{|c|}{3} & $\left(  a,\ora{e}\right)  $ & $\beta_{0}+\ora{\beta_{3}}$\\
\multicolumn{1}{|c|}{4} & $\left(  \bl{b},c\right)  $ & $\beta_{0}+\bl{\beta_{1}}$\\
\multicolumn{1}{|c|}{5} & $\left(  \bl{b},\ora{d}\right)  $ & $\beta_{0}+\bl{\beta_{1}}%
+\ora{\beta_{2}}$\\
\multicolumn{1}{|c|}{6} & $\left(  \bl{b},\ora{e}\right)  $ & $\beta_{0}+\bl{\beta_{1}}%
+\ora{\beta_{3}}$\\\hline
\end{tabular}
\end{center}

Vectors of binary variables corresponding to 6 rating cells:
\begin{align*}
\vec{x}_{1} &= (1,0,0,0) & \vec{x}_{4} &= (1,1,0,0)\\
\vec{x}_{2} &= (1,0,1,0) & \vec{x}_{5} &= (1,1,1,0) \\
\vec{x}_{3} &= (1,0,0,1) & \vec{x}_{6} &= (1,1,0,1)
\end{align*}
$X = [x_{ij}]_{i=1,j=1}^{4\,\,\,\,\,\,\,\,6}$ is the design matrix
\end{slidebox}\end{frame}

\begin{frame}\begin{slidebox}
\begin{itemize}
\item More generally,  for a rating factor with say $p$ categories,
we shall create $\left(  p-1\right)  $ binary variables, each of which takes
the value $1$ if tariff cell $i$ is in a given category and $0$, otherwise.
\pause
\item 
One of the categories is assumed to be the base category which is why we
create only $\left(  p-1\right)  $ binary variables. 
\end{itemize}
\end{slidebox}\end{frame}

\begin{frame}\begin{slidebox}
\mytitle{2.5.2 Link Function}
 \begin{tikzpicture}[domain=0:2]
\draw[blue, very thick] (0,0)  rectangle (14,6) ; 
\end{tikzpicture}
\end{slidebox}\end{frame}

\end{document}

\begin{frame}\begin{slidebox}
\end{slidebox}\end{frame}