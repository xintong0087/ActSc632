\documentclass[aspectratio=169,professionalfonts,mathserif,handout]{beamer}

\usepackage{pgf,tikz}
\usetikzlibrary{arrows,automata}
\usepackage{dsfont}
\usepackage{amsmath}
\usepackage{amssymb}
\usepackage{amsthm}
\usepackage{eulervm}
\usepackage{wasysym}
\usepackage[scaled]{helvet}
\usepackage[english]{babel}
\usepackage[T1]{fontenc}
\usepackage{hhline}
\usepackage{cancel}
\usepackage{comment}
\usepackage{booktabs}
\usepackage{multirow}
\usepackage{bm}
\usepackage{array}
\usepackage{ulem}
\usepackage[absolute,overlay]{textpos}
\usepackage{algorithm2e}



\usetikzlibrary{shapes,patterns,positioning,shapes.misc}
\renewcommand{\arraystretch}{1.2}
\frenchspacing

\newcommand\itemshape[1]{%
  \setbeamertemplate{itemize item}[#1]%
  \usebeamertemplate{itemize item}%
}

%-----------------------------------------------------------------------------%
% tikz
%-----------------------------------------------------------------------------%



\tikzset{cross/.style={cross out, draw=black, minimum size=2*(#1-\pgflinewidth), inner sep=0pt, outer sep=0pt},
%default radius will be 1pt. 
cross/.default={3pt}}


%-----------------------------------------------------------------------------%
% Colors
%-----------------------------------------------------------------------------%

% Black & White:

\definecolor{White}{rgb}{1,1,1}
\definecolor{LightGray}{rgb}{.91,.91,.91}
\definecolor{MediumGray}{rgb}{.75,.75,.75}
\definecolor{DarkGray}{rgb}{.4,.4,.4}
\definecolor{VeryDarkGray}{rgb}{.2,.2,.2}
\definecolor{Black}{rgb}{0,0,0}

% Purples:

\definecolor{RedPurple}{rgb}{.9,.0,.7}
\definecolor{LightPurple}{rgb}{.45,.0,.45}
\definecolor{DarkPurple}{rgb}{.1,.0,.3}
\definecolor{LightMagenta}{rgb}{0.95,.9,1}

% Blues:

\definecolor{Blue}{rgb}{0,0,0.9}
\definecolor{Navy}{rgb}{0,0,.5}
\definecolor{DarkBlue}{rgb}{0,0.1,0.3}

% Greens:

\definecolor{DarkGreen}{rgb}{.0,.2,.0}
\definecolor{Green}{rgb}{.0,.4,.0}

% Browns:

\definecolor{YellowBrown}{rgb}{1,1,0.7}
\definecolor{Cream}{rgb}{.9,.9,.75}
\definecolor{DarkerCream}{rgb}{.7,.7,.63}
\definecolor{Brown}{rgb}{0.2,0.25,0.1}
\definecolor{Coal}{rgb}{0.1,0.1,0.08}

% Other

\definecolor{CookieDough}{HTML}{ccccaa}
\definecolor{Clay}{HTML}{757561}
\definecolor{Licorice}{HTML}{21211c}
\definecolor{Red}{rgb}{0.8,0,0}

\colorlet{Back}{White}
\colorlet{Text}{Black}
\colorlet{TitleText}{DarkBlue}
\colorlet{Highlight}{DarkBlue}

\newcommand{\dg}[1]{{\color{Green}{#1}}}
\newcommand{\ora}[1]{{\color{orange}{#1}}}
\newcommand{\bl}[1]{{\color{Blue}{#1}}}
\newcommand{\red}[1]{{\color{Red}{#1}}}
\newcommand{\purple}[1]{{\color{DarkPurple}{#1}}}
\newcommand{\grey}[1]{{\color{MediumGray}{#1}}}

%\colorlet{Back}{Black!95}
%\colorlet{Text}{White}
%\colorlet{TitleText}{Blue}
%\colorlet{Highlight}{yellow}

%-----------------------------------------------------------------------------%
% Macros
%-----------------------------------------------------------------------------%

\newcommand{\tsp}{\mspace{1mu}}
\newcommand{\htsp}{\mspace{0.5mu}}
\newcommand{\op}[1]{\operatorname{#1}}
\newcommand{\tr}{\operatorname{Tr}}
\newcommand{\rank}{\operatorname{rank}}
\renewcommand{\t}{{\scriptscriptstyle\mathsf{T}}}
\newcommand{\reg}[1]{\mathsf{#1}}
\newcommand{\im}{\op{im}}
\newcommand{\fid}{\op{F}}
%\renewcommand{\vec}{\op{vec}}
\newcommand{\I}{\mathds{1}}

\newcommand{\abs}[1]{\lvert #1 \rvert}
\newcommand{\bigabs}[1]{\bigl\lvert #1 \bigr\rvert}
\newcommand{\Bigabs}[1]{\Bigl\lvert #1 \Bigr\rvert}
\newcommand{\biggabs}[1]{\biggl\lvert #1 \biggr\rvert}
\newcommand{\Biggabs}[1]{\Biggl\lvert #1 \Biggr\rvert}

\newcommand{\ip}[2]{\langle #1 , #2\rangle}
\newcommand{\bigip}[2]{\bigl\langle #1, #2 \bigr\rangle}
\newcommand{\Bigip}[2]{\Bigl\langle #1, #2 \Bigr\rangle}
\newcommand{\biggip}[2]{\biggl\langle #1, #2 \biggr\rangle}
\newcommand{\Biggip}[2]{\Biggl\langle #1, #2 \Biggr\rangle}

\newcommand{\ceil}[1]{\lceil #1 \rceil}
\newcommand{\bigceil}[1]{\bigl\lceil #1 \bigr\rceil}
\newcommand{\Bigceil}[1]{\Bigl\lceil #1 \Bigr\rceil}
\newcommand{\biggceil}[1]{\biggl\lceil #1 \biggr\rceil}
\newcommand{\Biggceil}[1]{\Biggl\lceil #1 \Biggr\rceil}

\newcommand{\floor}[1]{\lfloor #1 \rfloor}
\newcommand{\bigfloor}[1]{\bigl\lfloor #1 \bigr\rfloor}
\newcommand{\Bigfloor}[1]{\Bigl\lfloor #1 \Bigr\rfloor}
\newcommand{\biggfloor}[1]{\biggl\lfloor #1 \biggr\rfloor}
\newcommand{\Biggfloor}[1]{\Biggl\lfloor #1 \Biggr\rfloor}

\newcommand{\norm}[1]{\lVert\tsp #1 \tsp\rVert}
\newcommand{\bignorm}[1]{\bigl\lVert\tsp #1 \tsp\bigr\rVert}
\newcommand{\Bignorm}[1]{\Bigl\lVert\tsp #1 \tsp\Bigr\rVert}
\newcommand{\biggnorm}[1]{\biggl\lVert\tsp #1 \tsp\biggr\rVert}
\newcommand{\Biggnorm}[1]{\Biggl\lVert\tsp #1 \tsp\Biggr\rVert}




%\usepackage{amsfonts}
%\usepackage{graphicx}
%\usepackage{epsfig}
%\usepackage{subfigure}
%\usepackage{amssymb}% sophisticated mathematical symbols with amstex
%\usepackage{mathtools}% fix amsmath deficiencies
%\usepackage[T1]{fontenc}
%\usepackage{bm}% for bold math symbols
%\usepackage{bbm}% only for indicator functions
%\useoutertheme{default}
%\beamersetuncovermixins{\opaqueness<1>{25}}{\opaqueness<2->{15}}

\def\ba{{\mathbf a}}
\def\bd{{\mathbf d}}
%\def\bi{{\mathbf i}}
\def\bu{{\mathbf u}}
\def\bx{{\mathbf x}}
\def\bU{{\bf U}}
\def\bV{{\mathbf V}}
\def\bz{{\mathbf z}}
\def\bW{{\mathbf W}}
\def\bv{{\mathbf v}}
%\def\by{{\mathbf y}}
\def\bY{{\mathbf Y}}
\def\bw{{\mathbf w}}
\def\bX{{\mathbf X}}
\def\QMC{{\rm QMC}}
\def\bh{{\mathbf h}}
\def\MC{{\rm MC}}
\def\tra{{\mathrm T }}
\def\bz {\mathbf{z}}
\def\btpsi {\tilde{\boldsymbol{\psi}}}
\def\btv {\tilde{\mathbf{v}}}
\def\IR {\mathbb{R}}
\def\IP {\mathbb{P}}
\newcommand*{\IE}{\mathbbm{E}}
\newcommand*{\D}{\operatorname{D}}
\newcommand*{\Bcal}{\mathcal{B}}
\newcommand*{\Exp}{\operatorname{Exp}}
\newcommand*{\psii}{{\psi^{-1}}}
\newcommand*{\LS}{\mathcal{LS}}
\newcommand*{\LSi}{\LS^{-1}}

%\xdefinecolor{mypurple}{rgb}{0.6,0,0.7}
%\xdefinecolor{darkgreen}{rgb}{0,0.6,0}
%\xdefinecolor{MidYellow}{rgb}{1,0.5,0.2}

%\newcommand{\magenta}[1]{{\color{magenta}{#1}}}
%\newcommand{\gr}[1]{{\color{darkgreen}{#1}}}
%\newcommand{\MidYellow}[1]{{\color{MidYellow}{#1}}}
%\newcommand{\gray}[1]{{\color{gray}{#1}}}

\def\apause{\pause}
%\def\apause{}

\newenvironment{mylist}[1]{
  \begin{list}{}{
      \setlength{\leftmargin}{#1}
      \setlength{\rightmargin}{0mm}
      \setlength{\labelsep}{2mm}
      \setlength{\labelwidth}{8mm}
      \setlength{\itemsep}{0mm}}}{\end{list}}

\newcommand{\reference}[1]{\textcolor{Navy}{\footnotesize \sc #1}}

\newcommand{\highlight}[1]{\textcolor{Highlight}{\bf #1}}

%-----------------------------------------------------------------------------%
%Ruodu's stuff
%-----------------------------------------------------------------------------%


\def\fsd{\succcurlyeq_{\hspace{- 0.4 mm} _{fsd}}}
\def\sfsd{\succ_{\hspace{-0.4 mm} _{fsd}}}
\def\ssdmu{\succcurlyeq_{\hspace{- 0.4 mm} _{SSD\(\mu\)}}}
\def\ssdnu{\succcurlyeq_{\hspace{- 0.4 mm} _{SSD\(\nu\)}}}
\def\ssdP{\succcurlyeq_{\hspace{- 0.4 mm} _{SSD\(P\)}}}
\def\sssdmu{\succ_{\hspace{- 0.4 mm} _{SSD\(\mu\)}}}
\def\sssdnu{\succ_{\hspace{- 0.4 mm} _{SSD\(\nu\)}}}
\def\sssdP{\succ_{\hspace{- 0.4 mm} _{SSD\(P\)}}}
\def\Ito{It\^o}

\newcommand{\be}{\begin{eqnarray}}
\newcommand{\ee}{\end{eqnarray}}
\newcommand{\by}{\begin{eqnarray*}}
\newcommand{\ey}{\end{eqnarray*}}
\newcommand{\bi}{\begin{itemize}}
\newcommand{\ei}{\end{itemize}}

\DeclareMathOperator*{\argmax}{\arg\max}
\newtheorem{remark}[theorem]{Remark}
\newtheorem{assumption}[theorem]{Assumption}
\newtheorem{proposition}[theorem]{Proposition}

\setbeamercolor{yellow}{fg=black,bg=yellow}
\setbeamercolor{lightyellow}{fg=black,bg=yellow!40}
\setbeamercolor{orange}{fg=black,bg=orange}
\setbeamercolor{lightorange}{fg=black,bg=orange!40}
\setbeamercolor{lavender}{fg=black,bg= LavenderBlush!20!Lavender}

\newenvironment{colorblock}[2]
{\setbeamercolor{item}{fg=#1,bg=#1}\begin{beamerboxesrounded}[upper=#1,lower=#2,shadow=true]}
{\end{beamerboxesrounded}}

\definecolor{darkblue}{rgb}{0.1,0.1,0.6}
\definecolor{darkgreen}{rgb}{0.1,0.6,0.1}
\definecolor{fond}{RGB}{240,240,240}

\newcommand{\bluearrow}{\color{darkblue}{$\blacktriangleright$}\color{black}}
\newcommand{\redarrow}{\color{red}{$\blacktriangleright$}\color{black}}
\newcommand{\greenarrow}{\color{green}{$\blacktriangleright$}\color{black}}

\renewcommand{\(}{\left(}
\renewcommand{\)}{\right)}
\renewcommand{\[}{\left[}
\renewcommand{\]}{\right]}
\newcommand{\w}{\widehat}


%-----------------------------------------------------------------------------%
% Beamer stuff. Default theme -- everything needed is here.
%-----------------------------------------------------------------------------%

\usetheme{default}
\usefonttheme[onlymath]{serif}

% Aspect ratio is set to 16:9
% Total page width is 16cm

\setbeamersize{%
  text margin left=8mm,
  text margin right=8mm}

\addtolength{\headsep}{2.5mm}

\setbeamertemplate{navigation symbols}{}

\setbeamerfont{block title}{series=\bfseries}
\setbeamerfont{titlelike}{series=\bfseries}
\setbeamercolor{structure}{fg=Black}

\setbeamercolor*{titlelike}{fg=TitleText,bg=}
\setbeamercolor{normal text}{fg=Text,bg=Back}

\newcommand{\mytitle}[1]{\vspace*{-1mm}%
  \centerline{\highlight{\Large #1}}\vspace*{3mm}}

\newenvironment{slidebox}{%
  \begin{minipage}[c][7.5cm][t]{14.4cm}\raggedright}{%
  \end{minipage}}

\newenvironment{halfpage}{%
  \begin{minipage}[c][6.5cm][t]{6.4cm}\raggedright}{%
  \end{minipage}}

%-----------------------------------------------------------------------------%
\title{ACTSC 632 Week 4}
\author{Christiane Lemieux}
\date{\today}


\begin{document}


%-----------------------------------------------------------------------------%

\begin{frame}
  \begin{slidebox}
    
    \vspace{7mm}
    \begin{center}
      \begin{tikzpicture}

        \onslide<1>{%
        \node at (0,0) {%
          \makebox(0,0){\includegraphics[width=5.5cm]{UWlogo.eps}}};}

        \onslide<1>{%
        \node at (10.5,0) {%
          \makebox(0,0){ACTSC 632 Spring 2023%
            \rule[-3mm]{0mm}{6mm}
            \rule{6mm}{0mm}}};}
        
      \end{tikzpicture}
    \end{center}
    
    \vspace{6mm}
    
    \begin{center}
      \onslide<1>{%
      \highlight{\large Data Science with Actuarial Applications}\\[3mm]}
       \onslide<1>{\Huge {\color{darkblue} Week 7}}\\[6mm]
          \onslide<1>{\large %
        {\large Xintong Li}\\[3mm]
        Department of Statistics and Actuarial Science}
    \end{center}
    
\end{slidebox}\end{frame}

\begin{frame}\begin{slidebox}
\mytitle{Last Week}

\begin{itemize}
    \item $Y$ is a key ratio: claim frequency or claim severity
    \pause
    \item $X$ is a vector of rating factors, modeled as categorical variables 
    \pause
    \item Exponential Dispersion Models (EDM)
    \pause
    \item Multiplicative models (logarithmic link function)
    \pause
    \item Claim frequency: Poisson regression
    \pause
    \item Claim severity: Gamma regression
\end{itemize}
\end{slidebox}\end{frame}

\begin{frame}\begin{slidebox}
\mytitle{Today}

\begin{itemize}
    \item Estimating the parameters of a GLM from: 
    \pause
    \item Writing the log-likelihood function
    \pause
    \item Taking the derivative of the log-likelihood function
    \pause
    \item Solving for the maximum likelihood estimators (MLE)
\end{itemize}
\end{slidebox}\end{frame}

\begin{frame}\begin{slidebox}
\mytitle{2.6 Parameter Estimation (MLE)}
\begin{tikzpicture}[domain=0:2]
\draw[blue, very thick] (0,0)  rectangle (14,6) ; 
\end{tikzpicture}
\end{slidebox}\end{frame}

\begin{frame}\begin{slidebox}
\mytitle{Taking derivative using the chain rule}
 \begin{tikzpicture}[domain=0:2]
\draw[blue, very thick] (0,0)  rectangle (14,6) ; 
\end{tikzpicture}
\end{slidebox}\end{frame}

\begin{frame}\begin{slidebox}
\mytitle{Evaluating each partial derivative}
 \begin{tikzpicture}[domain=0:2]
\draw[blue, very thick] (0,0)  rectangle (14,6) ; 
\end{tikzpicture}
\end{slidebox}\end{frame}


\begin{frame}\begin{slidebox}
\mytitle{Remark: satured model}
 \begin{tikzpicture}[domain=0:2]
\draw[blue, very thick] (0,0)  rectangle (14,6) ; 
\end{tikzpicture}

Corresponding GLM overfits. However, it is useful in the definition of deviance.

\end{slidebox}\end{frame}

\begin{frame}\begin{slidebox}
\mytitle{Multiplicative Poisson frequency model}
 \begin{tikzpicture}[domain=0:2]
\draw[blue, very thick] (0,0)  rectangle (14,6) ; 
\end{tikzpicture}

\end{slidebox}\end{frame}


\begin{frame}\begin{slidebox}
\mytitle{Multiplicative gamma severity model}
 \begin{tikzpicture}[domain=0:2]
\draw[blue, very thick] (0,0)  rectangle (14,6) ; 
\end{tikzpicture}

\end{slidebox}\end{frame}


\begin{frame}\begin{slidebox}
\mytitle{Summary of Theoretical Results}
\begin{itemize}
    \item \highlight{Goal:} estimate the parameters of a GLM using MLE
    \pause
    \item The $(r+1)$ equations:
    \begin{equation*}
      \sum_{i=1}^{n}w_{i}\frac{y_{i}-\mu_{i}}{v\left(  \mu_{i}\right)  g^{\prime
      }\left(  \mu_{i}\right)  }x_{ij}=0\text{.} 
    \end{equation*}
    \pause
    \item For multiplicative Poisson frequency model:
    \begin{equation*}
      \sum_{i=1}^{n}w_{i} (y_{i}-\mu_{i}) x_{ij} = 0 \text{.}
    \end{equation*}  
    \pause
    \item For multiplicative gamma severity model:
    \begin{equation*}
      \sum_{i=1}^{n}w_{i}\frac{y_{i}-\mu_{i}}{\mu_{i}}x_{ij}=0\text{.} 
    \end{equation*}
\end{itemize}
\end{slidebox}\end{frame}

\begin{frame}\begin{slidebox}
  \mytitle{Example: Moped dataset}
  
  \begin{itemize}
      \item \highlight{Goal:} Use everything we learned to build a GLM for the moped dataset.
  \end{itemize}
  \end{slidebox}\end{frame}

\end{document}

\begin{frame}\begin{slidebox}
\end{slidebox}\end{frame}