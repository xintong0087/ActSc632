\documentclass[12pt]{article}

\usepackage[english]{babel}
\usepackage{inputenc}
\usepackage{amsmath}
\usepackage{amssymb}
\usepackage{graphicx}
\usepackage{wrapfig}
\usepackage{float}
\usepackage{subfigure}
\usepackage[margin=1in]{geometry}
\usepackage{pdfpages}
\usepackage{mathtools}
\usepackage{natbib}
\usepackage{bm}
\usepackage{multirow}
\DeclareMathOperator*{\argmax}{arg\,max}
\setlength\parindent{0pt}

\title{R Markdown Tutorial - Assignment}

\date{Spring 2023}

\begin{document}
\maketitle

\begin{enumerate}
    \item The Cauchy distribution is an example of a “heavy-tailed” distribution in that it will have outliers in both tails. This problem involves comparing it with a normal distribution which typically has very few outliers.
    \begin{enumerate}
        \item   Use \texttt{set.seed(124)} and \texttt{rcauchy()} with \texttt{n = 100, location = 0, scale = 0.1} to generate a random sample designated as y. \\
        Generate a second random sample designated as x with \texttt{set.seed(127)} and \texttt{rnorm()} using \texttt{n = 100, mean = 0, sd = 0.15}. Print their sample standard deviations. 
        \item   Plot their histogram side by side.
        \item   QQ plots are useful for detecting the presence of heavy tailed distributions. Using \texttt{qqnorm()} and \texttt{qqline()} to conduct QQ plots for both sets of samples. Add color and titles. Use \texttt{cex = 0.5} to control the size of the plotted data points.
    \end{enumerate}
\end{enumerate}
\end{document}