\documentclass[9pt,handout]{beamer}

\input{StyleFile.tex}


\title[\textbf{Efficient Nested Simulation with ML Proxy}]{Tutorial on R Markdown}
% \subtitle{\large \bfseries Machine Learning Proxies for High Dimensional Nested Simulation}

\author[\textbf{Xintong Li, xintong.li1@uwaterloo.ca}]
{\Large\bfseries
Xintong Li\\\medskip
xintong.li1@uwaterloo.ca
} 

\institute[\textbf{University of Waterloo, Actuarial Science}] % Your institution as it will appear on the bottom of every slide, may be shorthand to save space
{\large\bfseries
Dept. Statistics and Actuarial Science\\\smallskip
University of Waterloo % Your institution for the title page
}

\jointwork{ActSc 632}
\conference{June 2023}


%\usepackage[backend=bibtex,citestyle=authoryear-icomp,natbib,maxcitenames=1]{biblatex}
%\addbibresource{NestedSim.bib}

% use this so appendices' page numbers do not count
\usepackage{appendixnumberbeamer}
\usepackage{booktabs}
\usepackage{subfigure}
\usepackage[round]{natbib}
\usepackage{mathtools}


\usepackage{amsmath}
\usepackage{amsfonts}
\usepackage{amssymb}
\usepackage{bbm}
\usepackage{xcolor}
\usepackage{algorithm}
\usepackage{algpseudocode}
\usepackage{algorithmicx}
\usepackage{multirow}
\usepackage{graphicx}
\usepackage{fancyvrb}

\newcommand{\skipv}{\vspace{8pt}}


\begin{document}

% Title page, navigation surpressed, no page number
{
\beamertemplatenavigationsymbolsempty
\begin{frame}[plain]
\titlepage
\end{frame}
}



% TOC, navigation surpressed, no page number
% {
% \beamertemplatenavigationsymbolsempty
% \defbeamertemplate*{headline}{miniframes theme no subsection no content}
% { \begin{beamercolorbox}{section in head/foot}
%     \vskip\headheight
%   \end{beamercolorbox}}
% \begin{frame}{Outline}
% \tableofcontents
% \end{frame}
% }
\addtocounter{framenumber}{-1}

\begin{frame}{Introduction}

    R Markdown is an R package that can
    \skipv
    \begin{itemize}
        \item   create documents that combine R code, results, and text;
        \skipv
        \item   include plots, tables, and equations;
        \skipv
        \item   output documents in different formats, e.g. HTML, PDF, Word, etc.
    \end{itemize}
    \skipv
    \begin{block}{Installing and Using R Markdown in RStudio} 
        \quad \texttt{install.packages("rmarkdown")} \\ 
        \quad \texttt{library(rmarkdown)}
    \end{block}
    \skipv
    Thanks \cite{ourcodingclubGettingStarted} for creating and sharing this tutorial.

\end{frame}

\begin{frame}{Using R Markdown in RStudio}

    To create a new RMarkdown file (.Rmd) in RStudio,  
    \skipv
    \begin{enumerate}
        \item   select \texttt{File $\rightarrow$ New File $\rightarrow$ R Markdown};
        \skipv
        \item   choose the file type you want to create.
    \end{enumerate}
    \skipv
    For now we will focus on a .html document, which can be easily converted to other file types later.
    \begin{itemize}
        \item  .pdf documents can also be generated, but it may require additional software to be installed on your computer.
    \end{itemize}
    \begin{block}{Installing .pdf Support}
        \quad \texttt{install.packages("tinytex")} \\
        \quad \texttt{tinytex::install\_tinytex()} 
    \end{block}

\end{frame}

\begin{frame}{Title and Initial Knit}

    \begin{block}{Sample Title}
        \quad \texttt{---} \\ 
        \quad \texttt{title: "R Markdown Tutorial"} \\
        \quad \texttt{author: Xintong Li} \\ 
        \quad \texttt{date: 2/June/2023} \\ 
        \quad \texttt{output: html\_document} \\
        \quad \texttt{---} \\
    \end{block}
    \skipv

    \includegraphics[width=\textwidth]{Knit.jpeg}

\end{frame}

\begin{frame}{Formatting Text and Equations}

    We can use regular Markdown syntax in R Markdown to format text.
    \begin{itemize}
        \item   \texttt{*italics*}
        \item   \texttt{**bold**} 
        \item   \texttt{\#Header 1}
        \item   \texttt{\#\#Header 2}
        \item   \texttt{* Unordered list item} 
        \item   \texttt{1. Ordered list item}
    \end{itemize}
    \skipv
    We can also write equations using LaTeX syntax.
    \begin{block}{Sample Equation}
        Code \quad \texttt{\$A = \textbackslash pi \textbackslash times r\^ \quad \{2\}\$} results in:
        $$A = \pi \times r^{2}$$
    \end{block}

\end{frame}


\begin{frame}[fragile]{Including R Code Chunks}
    
    \begin{block}{Sample Code Chunk}
    \quad    \Verb|```{r, eval = FALSE, warning = FALSE}| \\
    \quad    \Verb|library(dplyr)| \\
    \quad    \Verb|A <- c("a", "a", "b", "b")| \\
    \quad    \Verb|B <- c(5, 10, 15, 20)| \\
    \quad    \Verb|dataframe <- data.frame(A, B)| \\
    \quad   \Verb|print(dataframe)| \\
    \quad    \Verb|```|
    \end{block}

    \skipv

    \begin{itemize}
        \item   \texttt{eval = FALSE} will not run the code, but will still display the code. 
        \item   \texttt{warning = FALSE} will suppress warnings when loading packages, etc. 
    \end{itemize}

\end{frame}

\begin{frame}{Including Plots and Tables}

    \begin{block}{A Sample Plot}
        \quad    \Verb|```\{r, fig.width = 4, fig.height = 3\}| \\
        \quad    \Verb|A <- c("a", "a", "b", "b")| \\
        \quad    \Verb|B <- c(5, 10, 15, 20)| \\
        \quad    \Verb|dataframe <- data.frame(A, B)| \\
        \quad    \Verb|print(dataframe)| \\
        \quad    \Verb|boxplot(B~A,data=dataframe)| \\
        \quad    \Verb|```|
    \end{block}
    \skipv
    \begin{block}{A Sample Table}
        \quad    \texttt{| Plant | Temp. | Growth |}\\
        \quad    \texttt{|:------|:-----:|-------:|}\\
        \quad    \texttt{| A     | 20    | 0.65   |}\\
        \quad    \texttt{| B     | 20    | 0.95   |}\\
        \quad    \texttt{| C     | 20    | 0.15   |}
    \end{block}
\end{frame}

\begin{frame}{Run Codes and Preview before Knitting}
    \begin{figure}
    \includegraphics[width=\textwidth]{Notebook_Run.jpg}
    \caption{Run Codes before Knitting}
    \end{figure}
    \begin{figure}
    \includegraphics[width=\textwidth]{Notebook_Preview.jpg}
    \caption{Generate a Preview}
    \end{figure}
\end{frame}

\begin{frame}[allowframebreaks]
    \frametitle{References}
    \bibliographystyle{apa}
    \bibliography{ref}
\end{frame}
\end{document}